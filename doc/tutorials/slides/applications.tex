\documentclass{beamer}
\usepackage{graphicx,url}
\usepackage{amsmath}
\usepackage{amssymb}
\usepackage{xkeyval}

\usepackage{mytikz}

\usepackage{basics}
\usepackage{basics-slides}

\newcommand{\lf}{\mathit{LF}}
\newcommand{\kity}{\mathit{type}}

\renewcommand{\emph}[1]{\alert{#1}}

\hypersetup{colorlinks}

\begin{document}

\title{MMT Tutorial, Part 2: \\ Application Development with MMT}
\author{Florian Rabe, Mihnea Iancu, Dennis M\"uller}
\institute{Jacobs University Bremen}
\date{CICM 2016}
\begin{frame}
    \titlepage 
\begin{center}
Bringing your notebook is recommended but not required.\\
Attending \hyperref{http://www.cicm-conference.org/2016/slides/MMTLanguages.pdf}{}{}{Part 1} is helpful but not required to follow Part 2.
\end{center}
\end{frame}


\section{Developing based on MMT}

\begin{myframe}{Users vs. Developers}
MMT blurs the distinction between users and developers
\begin{itemize}
 \item MMT is not an application itself
 \item It is a
  \begin{itemize}
    \item API for the MMT language
     \lec{close relative of OMDoc}
    \item suite of reusable algorithms/services
      \lec{e.g., MKM services}
    \item set of few example applications
       \lec{e.g., the IDE used in Part 1}
   \end{itemize}
 \item Primary users: developers of math applications
 \item Secondary users: users of those applications
\end{itemize}
\end{myframe}

\begin{myframe}{Basic Design of MMT Implementation}
\begin{itemize}
 \item Data structures for MMT language
  \lec{documents, modules, declarations, objects}
  \lec{get, add, update, delete}
 \item Core algorithms
    \lec{parse, check, simplify, present, \ldots}
    \lec{index, query, diff, build, \ldots}
 \item Backend
  \begin{itemize}
    \item catalog for mapping MMT URIs to physical locations
      \lec{in git repositories, databases, etc.}
    \item transparent loading/unloading into memory
  \end{itemize}
 \item Frontend
  \begin{itemize}
    \item API calls from Scala, Java
      \lec{programmatic or interactive}
    \item shell, scripting language
    \item HTTP server
      \lec{core algorithms exposed}
   \end{itemize}
\end{itemize}
\end{myframe}

\begin{myframe}{Extension Interfaces}
MMT systematically exposes extension interfaces
  \lec{essentially everything can be extended or replaced}
\begin{itemize}
 \item Customize core algorithms
   \begin{itemize}
    \item add rules
      \lec{declared in MMT theories}
    \item replace with custom implementations
    \item combine algorithms for structural and object level
   \end{itemize}
 \item Adding language features
   \begin{itemize}
     \item literals with external data/computation
     \item pragmatic features with elaboration semantics
   \end{itemize} 
 \item Import/export interfaces for integrating other formats and build targets
 \item Outside interface
  \begin{itemize}
    \item adding new command line syntax
    \item web framework for adding new HTTP interfaces
  \end{itemize}
 \item Change listening infrastructure for content events
\end{itemize}
\end{myframe}

\section{Breadth of Possible Applications}

% logic compiler for hets

\begin{myframe}{Example: jEdit IDE}
Application: IDE for MMT theories 
\begin{itemize}
  \item based on jEdit text editor
  \item induces IDE for any
    \begin{itemize}
    \item language defined in MMT
    \item language exported to OMDoc
    \end{itemize}
\end{itemize}

Design: MMT as p MMT and jEdit
\begin{itemize}
 \item Plugin for jEdit that wraps arond MMT
 \item MMT and jEdit large projects
 \item But only little glue code needed
   \lec{mostly forwarding to MMT API functions}
\end{itemize}
\end{myframe}

\begin{myframe}{Example: HOL Light library browser}
Based on
\begin{itemize}
 \item definition of HOL Light in MMT/LF
 \item export of HOL Light library in OMDoc format
\end{itemize}

MMT provides out of the box
  \begin{itemize}
    \item HTML+MathML rendering of library
      \lec{including 2-dimensional notations}
    \item as interactive documents
      \lec{definition lookup, type inference}
    \item dependency graph of theorems
   \end{itemize}
\end{myframe}

\begin{myframe}{Example: Generation of GAP Inheritance Graphs}
Based on GAP export to OMDoc

Only a few lines of code to
\begin{itemize}
 \item generate additional predicates for MMT's relational index
 \item register a new graph built from this predicate
\end{itemize}
yield
\begin{itemize}
 \item generation of inheritance graph between GAP filters
 \item interactive, integrated with web browser
\end{itemize}
\end{myframe}

\begin{myframe}{Example: Codec-Based Knowledge Exchange}
Problem:
\begin{itemize}
  \item mathematical data distributed over multiple databases
   \lec{LMFDb, OEIS, findStat.org, \ldots}
  \item in incompatible, unspecified low level encodings
\end{itemize}

Solution: Codec infrastructure of MMT
\begin{itemize}
 \item MMT theory for codecs and codec operators
 \item Describe database schemas 
   \begin{itemize}
    \item using mathematics-near types
    \item annotate codec expressions to define concrete encoding
   \end{itemize}
 \item Treat databases as MMT backends
\end{itemize}
Effect: MMT provides uniform high-level interface to low-level databases
\end{myframe}

\begin{myframe}{Example: MathHub}
Application:
\begin{itemize}
  \item Project hosting platform for mathematical library
    \lec{based on gitlab}
  \item Web interface for browsing, interacting with libraries
    \lec{based on Drupal}
  \item MMT used to process, interpret the content
\end{itemize}

Implementation: suite of MathHub-specific MMT extensions
\begin{itemize}
 \item new sTeX importer for interpreting sources
   \lec{uses MMT's build syste}
 \item HTML presenter for producing web pages
   \lec{uses MMT's MathML presenter for objects}
 \item HTTP server plugin for high-level MathHub specific queries
   \lec{uses MMT web framework, API functions}
 \item HTTP server plugin for browser-based editing
   \lec{uses same functions as jEdit IDE}
 \end{itemize}
\end{myframe}


\section{Structure of the Tutorial}

\begin{myframe}{Overview}
\begin{enumerate}
 \item Brief introduction to MMT-based Applications
 \item 3 mini-demos of prototypical MMT-based applications 
  \lec{easy for attendants to understand, reprocude, modify}
   \begin{enumerate}
    \item Changing equality by adding arbitrary rewrite or computation rules
    \item Using the MMT query interfaces to build a browser-based editor
    \item Using MMT's export infrastructure to build an OpenMath Content Dictionary editor
   \end{enumerate}
 \end{enumerate}
\end{myframe}


\begin{myframe}{Let's Start}
\begin{itemize}
 \item I will show mini-demos on the screen
 \item The tutorial points to the self-documenting source files
 \item Some demos can be run/changed by you as well
 \item Main link: \url{http://uniformal.github.io/doc/tutorials/applications/}
  \lec{no need to type this ---}
  \lec{these slides are linked from the CICM program}
\end{itemize}
\end{myframe}

\end{document}

